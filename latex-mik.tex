\documentclass[a4paper,draft,titlepage,abstracton]{scrartcl}
\setlength{\parskip}{1.1ex}
\setlength{\parindent}{0mm}
\usepackage[latin1]{inputenc}
\usepackage{textcomp}

\areaset[8mm]{17.0cm}{23.5cm}

\newcommand{\HRule}{\rule[1ex]{\linewidth}{0.7mm}}
\newcommand{\lm}{\LaTeX-Mik}

\usepackage{url}
\newcommand{\hoch}[1]{\raisebox{0.6ex}{#1}}
\newcommand{\tief}[1]{\raisebox{-0.6ex}{#1}}
\newcommand{\grad}{\ensuremath{^\circ}}
\newcommand{\pfeil}{\ensuremath{\rightarrow}}

\begin{document}

\begin{titlepage}
\raggedleft

\vspace*{\stretch{3}}

{ \Huge \sffamily
\lm: \LaTeX-Support for GVim \\on Win32-systems
\HRule \par}

{
 \Large  \sffamily
v0.7, September 22, 2005

}
{ \huge   \sffamily
Volker Kiefel
\par}



\vspace*{\stretch{8}}

\end{titlepage}

\begin{abstract}

Vim, one of the most complex text editors is also available on win32 systems.
Vim has excellent syntax highlighting for \LaTeX{} and Bib\TeX{} files. \lm{}
has been written for GVim, the graphical version of Vim. It supports authors
using Mik\TeX, a popular win32 \TeX/\LaTeX-implementation. Only minor changes
of the script are required to make it suitable for other \TeX{}
implementations, e.\,g. \texttt{fptex}.
\end{abstract}

\section{System requirements}

At least, installation of MikTeX\footnote{\url{http://wwww.miktex.org}} 
on a win32-system including \textbf{pdf\LaTeX} and
\textbf{dvips} is 
required. For converting \LaTeX-documents to HTML the script calls
\textbf{tth}\footnote{\url{http://hutchinson.belmont.ma.us/tth/}}. For
printing PostScript documents \textbf{Ghostview} and \textbf{Ghostscript}
should be installed.

\section{Installation}

Please copy the file \texttt{latex-mik.vim} into the \texttt{vim}
plugin-directory. If you open a file with the extension \texttt{.tex}
or\texttt{.bib}, the \LaTeX-menu appears.

\section{\LaTeX-menu options}

\subsection{Bib\TeX Entry}

\begin{sloppypar}
The Options of this menu insert empty templates for Bib\TeX{} document types
into a \texttt{.bib} file\footnote{A Bib\TeX{} database usually has the
filename extension \texttt{.bib}}. Thus \texttt{LaTeX\pfeil BibTeX\pfeil
Article} inserts

\end{sloppypar}

\begin{verbatim}
  @article{,
  author = {},
  title = {},
  year = {},
  journal = {},
  OPTpages = {},
  OPTvolume = {},
  OPTmonth = {},
  OPTnumber = {}
  }
\end{verbatim}

into the text: the template for \verb+@article+-entries. Fields beginning with
\texttt{OPT...} are optional in the context of this document type. The last
field of a well-formed \texttt{.bib}-file is not closed by a comma (,). Also
the other document types: book, booklet, inbook, incollection, inproceedings,
manual, mastersthesis, misc, phdthesis, proceedings, techreport, unpublished
are implemented.

\subsection{Environment on region, Empty environment}

\texttt{LaTeX\pfeil Environment on region} only works if a region is selected
``linewise visual'' (with `V' in Vim's command mode). With a region selected
``characterwise visual'' (`v') the script generates the error message: ``No
text highlighted linewise''. To use this command, select a region of lines
with `V' and enter the name of the environment ant the prompt, e.\,g.
\texttt{center}. \texttt{LaTeX\pfeil Empty environment} generates an empty
environment. \lm{} attempts to create ``appropriate'' environment templates.
As an example, ``\texttt{center}'' inserts

\begin{verbatim}
  \begin{center}
  
  \end{center}
\end{verbatim}

``\texttt{itemize}'' inserts

\begin{verbatim}
  \begin{itemize}
    \item 
  \end{itemize}
\end{verbatim}

and ``\texttt{table}'' inserts

\begin{verbatim}
  \begin{table}[]
  
    \caption[]{}
    \label{}
  \end{table}
\end{verbatim}


\subsection{Commands on region, Empty Commands}

\texttt{LaTeX\pfeil Commands on region} only works if a region is selected
``characterwise visual'', with a region selected ``linewise visual'', the
script generates the message ``No text highlighted characterwise''.
\texttt{LaTeX\pfeil Empty commands} generates an empty command.


\subsection{``Umlaut'' and accented characters}

\begin{sloppypar}
In \LaTeX-documents ``Umlaut''-characters (������) and `�' are inserted
literally by default. If you select \texttt{LaTeX\pfeil Umlaute\pfeil Normal
TeX}, typing \texttt{�} inserts \verb+\"a+. Default behaviour is restored with
\texttt{LaTeX\pfeil Umlaute\pfeil Normal TeX}. 
\texttt{LaTeX\pfeil Umlaute\pfeil German TeX mapping} prints \verb+"a+ for a
typed \verb+�+, \texttt{LaTeX\pfeil Umlaute\pfeil BibTeX mapping} prints 
\verb+{\"a}+, 
\texttt{LaTeX\pfeil Umlaute\pfeil German Umlaut mapping}
converts \verb+�+ to \verb+ae+. Mapping for accented and other
characters might easily
be added.
\end{sloppypar}



\subsection{Process \LaTeX-projects}

In the following description, a \LaTeX-project is understood either as a
single \LaTeX-file (together with a \texttt{.bib}-file, if it exists) or as
the main file and the files included with \verb+\input{}+ or \verb+\include{}+
commands as in

\begin{verbatim}
   \documentclass{article}
   \begin{document}

   \input{fileOne}
   \input{fileTwo}
   \input{fileThree}

   \end{document}
\end{verbatim}

After opening one of the files of a project and before \LaTeX ing and/or viewing
the project you will have to tell the script the project name, therefore
select \texttt{LaTeX\pfeil Projectname}. If you are currently editing the main
file, you may confirm the prompt (\texttt{Enter project name [default current
  file]:}) with [Enter], if another file of the project is edited you will
have to enter the name of the main file without the extension\footnote{if the
  main file is \texttt{myfile.tex}, please enter \texttt{myfile}; generally the
  main file is the \TeX{} file with the \texttt{\textbackslash 
  documentclass{}} statement}. 
Then, with \texttt{LaTeX\pfeil LaTeXProject} you will compile the project
(produce a \texttt{.dvi}-file), 
\texttt{LaTeX\pfeil BibTeXProject} runs Bib\TeX,
\texttt{LaTeX\pfeil IndexProject} runs \texttt{makeindex},
\texttt{LaTeX\pfeil ViewFile} calls the dvi-viewer \texttt{yap} and makes it
jump to the correct position,
\texttt{LaTeX\pfeil PDFLaTeX} calls PDF\LaTeX\ to produce a PDF-file,
\texttt{LaTeX\pfeil dvips} calls \texttt{dvips} to produce a PostScript file,
\texttt{LaTeX\pfeil gsview} opens the grahical interface of the
ghostscript-interpreter, 
\texttt{LaTeX\pfeil LaTeX to HTML} generates a HTML file with
tth\footnote{available through \url{http://hutchinson.belmont.ma.us/tth/}}. 

\subsection{Mappings}

Important funktions have been mapped to key sequences beginnung with
``,''. 

\begin{table}[h]

\begin{center}
\begin{tabular}{|ll|} \hline
\textbf{Command} & \textbf{mapping} \\ \hline
\texttt{\LaTeX Project} & \texttt{,la} \\
\texttt{ViewFile} & \texttt{,vi} \\ 
\texttt{Projectname} &  \texttt{,pr} \\
\texttt{Empty Commands} & \texttt{,cm} \\
\texttt{Commands on region} & \texttt{,rcm} \\
\texttt{Empty environment} & \texttt{,en} \\
\texttt{Environment on region} & \texttt{,ren} \\
 \hline
\end{tabular}
\end{center}
  \caption{Mappings of \lm-functions}
  \label{maps}
\end{table}

For example, if you wish to \LaTeX{}
the current project you may enter \texttt{,la} instead of selecting
the menu option
\texttt{\LaTeX Project}. All current mappings are listed in table \ref{maps}.

\end{document}

